\chapter{Metodología}


Para lograr cada objetivo específico propuesto para este trabajo se propone un desarrollo de cinco etapas donde la finalización de cada etapa cumplirá con un objetivo específico.

\section{Reconocimiento teórico}
 Se inicia con una exploración a la literatura académica de publicaciones relacionada con alineadores forzados y sus aplicaciones para mejorar el nivel de anotación de corpus hablados, estudiando los principios propuestos, implementación y relevancia en el presente. Para esta etapa se estiman dos meses de trabajo.
 
 Se estudiarán aplicaciones teóricas así como implementaciones publicadas bajo licencia abierta, estas implementaciones serán estudiadas y clasificadas según la técnica de alineación utilizada

\section{Recolección de recursos de licencia abierta}
Finalizada la etapa de reconocimiento teórico de herramientas se recolectarán recursos hablados de licencias abiertas anotados y no anotados, que cumplan características apropiadas para ser procesados, incluyendo que sean de único locutor, su dicción sea clara y no estén en ambientes ruidosos. 

En caso de ser recursos anotados a nivel de palabras o fonético, se evaluará la anotación respectiva con las implementaciones abiertas estudiadas.

También se caracterizarán los recursos abiertos en función de su duración, el nivel de anotación y cantidad de locutores.

Si la licencia de los corpus lo permite, al final de esta etapa se tendrá de igual manera un repositorio espejo con los corpus abiertos para un fácil acceso, en los casos donde no sea posible alojar públicamente los datos, se dejará un enlace a la fuente autorizada para las posteriores descargas.

\section{Construcción de un corpus de pruebas}
Para evaluar el desempeño del alineador forzado que se construirá y utilizando los recursos abiertos y anotados a nivel fonético, se diseñará un corpus de prueba fonéticamente balanceado para la lengua española. Este corpus de prueba constará de recursos anotados existentes y de licencia abierta así como un conjunto de prueba anotado manualmente para la verificación, considerando situaciones como el cambio de ruido en el ambiente y alófonos.

Este corpus será liberado bajo licencias abiertas.

\section{Implementación del alineador forzado}
Con el análisis de las herramientas para la anotación automática de recursos hablados, se dará inicio a la implentación de un alineador forzado para la lengua española, tomando como base las aproximaciones teóricas encontradas en la primera fase, utilizando metodologías iterativas, buscando en primera instancia el la segmentación de frases de un único locutor a nivel de palabras, seguido por la segmentación a nivel de palabras independiente de locutor. 

Esta etapa dejará código fuente con la implementación del alineador diseñado, dejando el trabajo alojado en fuentes públicas para el trabajo e investigación posterior.


\section{Diseño de pruebas de nivel de anotación}
Finalmente se diseñarán pruebas empíricas para determinar la precisión y eficacia del anotador, indicando su rendimiento en un nuevo corpus de gran vocabulario para la lengua española de licencia abierta.

La anotación realizada al corpus se publicará bajo licencias abiertas para su posterior uso en otras investigaciones.