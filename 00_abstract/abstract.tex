\addcontentsline{toc}{chapter}{Abstract}

\begin{abstract}
El reconocimiento automático del habla (Automatic Speech Recognition ASR) es una subarea de la inteligencia artificial con un amplio campo de investigación en la actualidad. Muchos avances tecnológicos han sido mostrados en la industria soportados por investigación académica enfocados en el idioma inglés, pero replicar los experimentos en otros idiomas presenta problemáticas por la falta de corpus apropiados. 

% Automatic Speech Recognition is a giant subfield of Artificial Intelligence and a broad topic for research now. Many new advances have been shown in industry supported by academic research, but replicate experiments in languages different than the English language could be a problem for the lack of proper corpora. 

La lengua española cuenta con recursos de licencia abierta hablados y anotados, pero estos recursos no son adecuados ni suficientes para replicar los resultados de investigación obtenidos para el idioma inglés. Esta tesis busca crear un corpus hablado y anotado usando recursos abiertos existentes para replicar resultados obtenidos en lenguajes extranjeros.
% The Spanish language has  speech annotated data, but its annotation is not adequate to replicate existing research with the similar results. This thesis aims to create a fine grained annotated corpora for the Spanish language to replicate existing research in foreigns languages.
% Text of the Abstract.

% Información clara y concisa de la idea del proyecto.
% • Exponga brevemente los antecedentes, el problema y la solución propuesta.
% • Exponga brevemente los resultados esperados y el impacto de ellos.
% • Lo novedoso de su propuesta y las perspectivas de trabajo futuro que le permitirá explorar la propuesta.

\end{abstract}